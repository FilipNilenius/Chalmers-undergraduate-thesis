\chapter{Introduction}
This is a template! 


\section{Background}
Write about the thesis background here.
\section{Problem description}
This is where you describe you problem.
\section{General aim}
By now you know what to do here :)
\section{Method / Outline }
...

\section{Objectives}
...

\section{References}
Your reference data should be contained in \texttt{referencedata.bib}. Open the file using a text editor and look at its content. Your own references need to have the same structure! You cite a reference in this way: \textcite{Harryson2014} and \textcite{Noren2006} \citeauthor{Noren2006}.

\section{Cross-references}
Cross-references within your own thesis are taken care of by the package \texttt{cleverref}. Making a cross-reference to a figure is done in this way: \cref{figure:test} (the name in the curly brackets could be anything).

\section{Equations}
Here is how to typeset equations in \LaTeX{}.

\begin{align}
\sigma = E\varepsilon \label{coolequation}
\end{align}
and here is how to align several equations using the \& symbol:
\begin{align}
A &= Bx \\ % double backslash to break to a new line
c + D +\frac{2}{\phi} &= \sqrt{B}
\end{align}
and here is how to suppress numbering of equations
\begin{align*}
\sigma = E\varepsilon
\end{align*}
and this is how to cross-reference to an equation: \cref{coolequation}.

\subsection{In-line math}
Use the \$-symbol to typeset in-line math like so: $\sigma = E\varepsilon$. This important because in-line math should be italicized. For example, if you want to write the symbol for Young's modulus, it needs to be done in this way: $E$, not: E. If the letter ``E'' is italicized, then it is a physical quantity, namely Young's modulus, whereas a normal ``E'' is just an E.